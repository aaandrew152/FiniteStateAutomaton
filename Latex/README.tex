\documentclass[11pt,twoside]{report}

\usepackage{graphicx}
\usepackage{amsmath}
\usepackage{hyperref}
\usepackage{amssymb}
\usepackage{enumerate}

\hypersetup{colorlinks=true, linkcolor=black, urlcolor=blue}

\setlength{\topmargin}{-1.0in}
\setlength{\textheight}{9.25in}
\setlength{\oddsidemargin}{0.0in}
\setlength{\evensidemargin}{0.0in}
\setlength{\textwidth}{6.5in}
\newenvironment{proof}[1][Proof]{\noindent\textbf{#1.} }{\ \rule{0.5em}{0.5em}}
\setlength{\parindent}{0in}

\begin{document}

\begin{center}
\section*{FiniteStateAutomaton}

Code Base by Andrew Ferdowsian\\*
White Paper + Updates by Justin Oliver\\*
May 13, 2016
\end{center}

\subsection*{GitHub}

\url{https://github.com/aaandrew152/FiniteStateAutomaton}

\subsection*{Requirements}

The graphical output of the library depends on \texttt{matplotlib} and \texttt{networkx}. The command line graphic also requires the \texttt{progress} library. All libraries can be installed using \texttt{\$ pip install ...}

\subsection*{Usage}

The desired simulation should be defined in \texttt{Parameters.py} using the guidelines outlined below. Once the simulation has been set up, simply navigate to the root directory and use \texttt{\$ python main.py} to execute the simulation. If you wish to run the same simulation multiple times with one function call, use \texttt{\$ python master.py} instead. Results will be generated according to the parameters specified and output in the desired format. 

\subsection*{Parameters}

\subsubsection*{Simulation Parameters}

\begin{itemize}
\item \texttt{numSets}: the number of sets of individuals in the population 
\item \texttt{numGenerations}: the number of generations the simulation will be run for 
\item \texttt{collectGenerations}: the number of generations before the final generations to begin recording all actions played by individuals
\begin{itemize}
\item must be less than or equal to the total number of generations
\item set to zero if this function is not desired
\end{itemize}
\item \texttt{numSims}: the total number of simulations to be run (only relevant when using \texttt{master.py})
\end{itemize}

\subsubsection*{Game Parameters}

\begin{itemize}
\item \texttt{payoffMatrix}: the game's payoff matrix, i.e. ((2,0),(3,1)) for Prisoner's Dilemma  
\item \texttt{discountFactor}: the probability $\delta$ that a game is repeated (i.e. does not end) within a given generation
\item \texttt{mutationProb}: the probability that a given individual undergoes mutation at the end of a generation, with probabilities of specific mutation types specified below
\begin{itemize}
\item \texttt{mutation\_addState}: given an individual is mutated, probability that a new state is added (random action chosen for the state, arrow randomly chosen from all existing arrows is reassigned to point to this new state)
\item \texttt{mutation\_deleteState}: given an individual is mutated, probability that an existing state is deleted (all arrows currently pointing to this state are randomly reassigned to point to another state)
\item \texttt{mutation\_changeArrow}: given an individual is mutated, probability that an existing arrow is randomly reassigned to a new target
\item \texttt{mutation\_changeAction}: given an individual is mutated, probability that the action at an existing state is randomly reassigned (this probability cannot be specified, but is rather $1 - \sum (\text{previous probabilities})$)
\end{itemize}
\item \texttt{startingStrategyDistribution}: the distribution of state actions at the first generation, i.e. [0.5,0.5] would state that half of all individuals begin playing each action
\item \texttt{noise}: is there noise in the system (TO BE IMPLEMENTED)
\item \texttt{mutationPrune}: if set to \texttt{True}, an individual's strategy will be reduced such that all states are accessible from the origin state \textit{any} time a mutation occurs, if \texttt{False} this reduction will only occur at the final generation 
\end{itemize}

\subsubsection*{Output Parameters}

\begin{itemize}
\item \texttt{prevalence}: the threshold of prevalence in the population for a strategy to be recorded, all strategies with prevalences below this threshold will essentially be thrown out at the final generation, set to 0 if all strategies should be saved
\item \texttt{saveOutput}: if \texttt{True} simulation results will be saved to a .txt file in a directory specified below, if \texttt{False} results will be printed to standard output
\item \texttt{withGraphics}: if \texttt{True} graphics representing each strategy will be generated and saved in the appropriate directory
\item \texttt{directory}: specify the directory for the output to be saved into, relative to the current location of the root folder
\begin{itemize}
\item if left as \texttt{None}, the output will be saved in a subdirectory entitled \texttt{simulations/}
\item if the directory already exists, \textit{it will be erased and replaced by the new results}
\item the specified directory will contain subfolders for each simulation run 
\end{itemize}
\end{itemize}

\subsection*{Output}

\subsubsection*{Summary}

When simulation results are set to be saved (see \texttt{saveOutput} parameter), a file called `summary.txt' will be generated in the specified directory. Otherwise the results will be printed to the standard output. These results contain:
\begin{itemize}
\item a summary of the parameters specified for the simulation
\item the average amount all given actions were played by all individuals over the specified number of generations (see \texttt{collectGenerations} parameter)
\item a list of all strategies present at the final generation and their prevalences
\end{itemize}

Each strategy in the list is a textual representation of the FiniteStateAutomaton object. Each item in the list represents a state as a tuple. Each state contains the following information (in order):
\begin{enumerate}
\item the number of the state (0 is the origin state from which all strategies begin)
\item the action to be played at the state
\item a list of arrows pointing away from the state, each of which is listed as a tuple of the form (target state, condition)
\end{enumerate}

\subsubsection*{Graphics}

When graphics are generated (see \texttt{withGraphics} parameter), a .png image will be generated for each strategy present at the final generation. The images are saved with the title `Strategy\#\_prevalance\%.png'. The graphics can be understood as follows:
\begin{itemize}
\item each state is a circle with the action at that state encoded as a color
\item each arrow is a black line where the head of the arrow is thicker
\item each arrow is labeled with the condition of its use, represented as the action (as a number) required to take that particular path
\item the origin state (0) is marked with *
\end{itemize}

\end{document}